\documentclass[12pt]{article}
    \title{\vspace{-50pt}\textbf{\underline{CAP6635 Assignment 2 - Problem 1}}}
    \author{William L. Thomson Jr.}
    \date{Spring Semester 2024}

    \addtolength{\topmargin}{-3cm}
    \addtolength{\textheight}{3cm}
    
    \usepackage{enumitem}
\begin{document}

\maketitle

\begin{description}[itemsep=1em]
	\item[What does a population represent in this case?] \hfill \break
		All of the various puzzle states, every possible N-puzzle position, all puzzle state
		position permutations, including goal state; An array or string of unique numbers
		representing the value in each position. \\
		E.g. for 8-puzzle goal 012345678.

        However, if applying a Monte Carlo method, where you are analyzing a sequence
        of actions, then that could be used as the population, all the possible action step
        combinations to solving a puzzle based on a given start state, to find one or more
        goal sequences. This would need some cut off point so that all sequences of
        actions are the same length. This would be to experiment with the possibility 
        of solving with a fixed and limited number of actions. \\
		E.g. for 8 action 8-puzzle starting/ending in empty 0,0 LLUURRDD.

        The population could be either all the various states, and seeing which states
        can lead to an order of actions to follow, or using the actions as the population
        to determine the way to solve a given state.

	\item[What is your mutation function?] \hfill \break
		A function that changes the various values representing the tile positions in any puzzle
		state, choosing a random position to change, within the rules of the game. That would
		entail replacing the new value with the old at what ever position the old was located,
		as game rules would not allow for the same value in multiple locations, nor absence
		of others. \\
		E.g. for 8-puzzle 8765\underline{4}3210 $\rightarrow$ 4765\underline{8}3210

        This is basically the same as the generate function used for Simulated Annealing,
        where you would pick a random position to change/mutate, which is like generating
        a new puzzle state in SA. Both require that as you change one, you must change
        another as values cannot be duplicated, puzzle pieces are unique. It is possible
        the exact generate function could be used for both GA and SA.

	\item[How does crossover happen in this case?] \hfill \break
		Cross over would need to take place based on the max/min amount of unique characters
		in each puzzle state, as the crossover could not duplicate the values for any puzzle
		state. This would vary between any two states being crossed over, but should be at
        at least 2-3 unique in each, but its possible there could be less. \\
		E.g. for 8-puzzle \underline{012}345678 x 876543\underline{210} $\rightarrow$ \underline{210}345678 | 876543\underline{012}

	\item[How do you evaluate the quality of a chromosome in this scenario?] \hfill
		You would need some scoring mechanism to score the resulting game state, aka
		chromosome, to determine which ones are of highest value, highest score, closest to
		or the goal state. One method of scoring combines the number of correct puzzle
		positions along with the Manhattan distance, with lower score being more desirable
		and goal/stop at 0 to formulate the heuristic.

        This would most likely be the exact same heuristics and scoring mechanism
        as used in Simulated Annealing, as both would need to score the puzzle
        state the same.

\end{description}
\end{document}

